\documentclass[11pt,fleqn]{article}

\setlength {\topmargin} {-.15in}
\setlength {\textheight} {8.6in}

\usepackage{amsmath}
\usepackage{amssymb}
\usepackage{amsthm}
\usepackage{color}
\usepackage[margin=0.7in]{geometry}

\renewcommand{\labelenumi}{\theenumi.}
\renewcommand{\labelenumii}{\theenumii.}
\renewcommand{\labelenumiii}{\theenumiii.}
\newcommand{\be}{\begin{enumerate}}
\newcommand{\ee}{\end{enumerate}}
\newcommand{\bi}{\begin{itemize}}
\newcommand{\ei}{\end{itemize}}
\newcommand{\bc}{\begin{center}}
\newcommand{\ec}{\end{center}}
\newcommand{\bsp}{\begin{sloppypar}}
\newcommand{\esp}{\end{sloppypar}}
\newcommand{\mname}[1]{\mbox{\sf #1}}
\newcommand{\pnote}[1]{{\langle \text{#1} \rangle}}
\newcommand{\sB}{\mbox{$\cal B$}}
\newcommand{\sC}{\mbox{$\cal C$}}
\newcommand{\sF}{\mbox{$\cal F$}}
\newcommand{\sP}{\mbox{$\cal P$}}
\newcommand{\subfun}{\sqsubseteq}
\ifdefined \And 
\renewcommand{\And}{\wedge}
\else
\newcommand{\And}{\wedge}
\fi
\newcommand{\Or}{\vee}
\newcommand{\Implies}{\Rightarrow}
\newcommand{\Iff}{\LeftRightarrow}
\newcommand{\Forall}{\forall}
\newcommand{\ForallApp}{\forall\,}
\newcommand{\Forsome}{\exists}
\newcommand{\ForsomeApp}{\exists\,}
\newcommand{\mdot}{\mathrel.}
\newcommand{\set}[1]{{\{ #1 \}}}

\begin{document}

\begin{center}
	
	{\large \textbf{COMPSCI/SFWRENG 2FA3}}\\[2mm]
	{\large \textbf{Discrete Mathematics with Applications II}}\\[2mm]
	{\large \textbf{Winter 2021}}\\[8mm]
	{\huge \textbf{Assignment 3}}\\[6mm]
	{\large \textbf{Dr.~William M. Farmer and Dr.~Mehrnoosh Askarpour}}\\[2mm]
	{\large \textbf{McMaster University}}\\[6mm]
	{\large Revised: February 3, 2021}
	
\end{center}

\medskip

Assignment 1 consists of some background definitions, two sample
problems, and two required problems.  You must write your solutions to
the required problems using LaTeX.  Use the solutions of the sample
problems as a guide.

Please submit Assignment~1 as two files,
\texttt{Assignment\_1\_\emph{YourMacID}.tex} and
\texttt{Assignment\_1\_\emph{YourMacID}.pdf}, to the Assignment~1
folder on Avenue under Assessments/Assignments.
\texttt{\emph{YourMacID}} must be your personal MacID (written without
capitalization).  The \texttt{Assignment\_1\_\emph{YourMacID}.tex}
file is a copy of the LaTeX source file for this assignment
(\texttt{Assignment\_1.tex} found on Avenue under
Contents/Assignments) with your solution entered after each required
problem.  The \texttt{Assignment\_1\_\emph{YourMacID}.pdf} is the PDF
output produced by executing

\begin{itemize}
	
	\item[] \texttt{pdflatex Assignment\_1\_\emph{YourMacID}}
	
\end{itemize}

This assignment is due \textbf{Sunday, February 14, 2021 before
	midnight.}  You are allow to submit the assignment multiple times,
but only the last submission will be marked.  \textbf{Late submissions
	and files that are not named exactly as specified above will not be
	accepted!}  It is suggested that you submit your preliminary
\texttt{Assignment\_1\_\emph{YourMacID}.tex} and
\texttt{Assignment\_1\_\emph{YourMacID}.pdf} files well before the
deadline so that your mark is not zero if, e.g., your computer fails
at 11:50 PM on February 14.

\textbf{Although you are allowed to receive help from the
  instructional staff and other students, your submission must be your
  own work.  Copying will be treated as academic dishonesty! If any of
  the ideas used in your submission were obtained from other students
  or sources outside of the lectures and tutorials, you must
  acknowledge where or from whom these ideas were obtained.}

\subsection*{Background}

A \emph{word} over an alphabet $\Sigma$ of symbols is a string
\[a_1a_2a_3{\cdots}a_n\] of symbols from $\Sigma$.  For example, if $\Sigma
= \set{a,b,c}$, then the following are words over $\Sigma$ among many
others: \bi

  \item $cbaca$.

  \item $ba$.

  \item $acbbca$.

  \item $a$

  \item $\epsilon$ (which denotes the empty word).

\ei 

Let $\Sigma^*$ be the set of all words over $\Sigma$ (which includes
$\epsilon$, the empty word).  Associated with each word $w \in
\Sigma^*$ is a set of positions.  For example, $\set{1,2,3}$ is set of
positions of the word $abc$ with the symbol $a$ occupying position 1,
$b$ occupying position 2, and $c$ occupying position 3.  If $u,v \in
\Sigma^*$, $uv$ is the word in $\Sigma^*$ that results from
concatenating $u$ and $v$.  For example, if $u = aba$ and $v = bba$,
then $uv = ababba$.

A \emph{language} $L$ over $\Sigma$ is a subset of $\Sigma^*$.  A
language can be specified by a first-order formula in which the
quantifiers range over the set of positions in a word.  In order to
write such formulas we need some predicates on positions.
$\mname{last}(x)$ asserts that position $x$ is the last position in a
word.  For $a \in \Sigma$, $a(x)$ asserts that the symbol $a$ occupies
position $x$.  For example, the formula \[\ForallApp x \mdot
\mname{last}(x) \rightarrow a(x)\] says the symbol $a$ occupies the
last position of a word.  This formula is true, e.g., for the words
$aaa$, $a$, and $bca$.

The language over $\Sigma$ specified by a formula is the set of words
in $\Sigma^*$ for which the formula is true.  For example, if $a \in
\Sigma$, then $\ForallApp x \mdot \mname{last}(x) \rightarrow a(x)$ specifies
the language $\set{ua \mid u \in \Sigma^*}$.

\subsection*{Problems}

\be

  \item \textbf{[12 points]} Let $\Sigma$ be a finite alphabet and
    $\Sigma^*$ be the set of words over $\Sigma$.  Define $u \le v$ to
    be mean there are $x,y \in \Sigma^*$ such that $xuy = v$.  That
    is, $u \le v$ holds iff $u$ is a subword of $v$.

    \textbf{Aamina Hussain, hussaa54, February 14, 2021}

  \be

    \item Prove that $(\Sigma^*,\le)$ is a weak partial order.

    \emph{Proof.} To prove that $(\Sigma^*,\le)$ is a weak partial order, we must prove it is reflexive, antisymmetric, and transitive.\\
    \\
    \emph{Reflexive:} $\ForallApp u \in \Sigma^* \mdot \exists x,y \in \Sigma^* \mdot u \le u$

    \begin{flalign*}
      &\mname{For all } u \in \Sigma^*:\\
      &\phantom{{}=} \exists x,y \in \Sigma^* \mdot u \le u\\
      &= \exists x,y \in \Sigma^* \mdot u \le xuy  & \pnote{def. of a subword}\\
      &= u \le xuy  & \pnote{$\exists-introduction$; x,y = x,y}\\
      &= true & \pnote{def. of a subword}\\
    \end{flalign*}

    So reflexivity of $(\Sigma^*,\le)$ holds.\\


    \emph{Antisymmetric:} $\ForallApp u, v \in \Sigma^* \mdot \exists x,y \in \Sigma^* \mdot u \le v \land v \le u \implies u = v$

    \begin{flalign*}
      &\mname{For all } u, v \in \Sigma^*:\\
      &\phantom{{}=} \exists x,y \in \Sigma^* \mdot u \le v \land v \le u \implies u = v\\
      &= \exists x,y \in \Sigma^* \mdot u \le xuy \land xuy \le u \implies u = xuy & \pnote{def. of a subword; v=xuy}\\
      &= u \le \epsilon u \epsilon \land \epsilon u\epsilon \le u \implies u = \epsilon u\epsilon  & \pnote{$\exists-introduction$; x,y = $\epsilon,\epsilon$}\\
      &= u \le u \land u \le u \implies u=u & \pnote{empty word}\\
      &= true \land true \implies true & \pnote{def. of a subword}\\
      &= true\\
    \end{flalign*}

    So antisymmetry of $(\Sigma^*,\le)$ holds.\\

    \emph{Transitive:} $\ForallApp u, v, w \in \Sigma^* \mdot \exists a, b, c, d \in \Sigma^* \mdot u \le v \land v \le w \implies u = w$

    \begin{flalign*}
      &\mname{For all } u, v, w \in \Sigma^*:\\
      &\phantom{{}=} \exists a,b,c,d \in \Sigma^* \mdot u \le v \land v \le w \implies u \le w\\
      &= \exists a,b,c,d \in \Sigma^* \mdot u \le v \land v \le avb \implies u \le avb & \pnote{def. of a subword; w=avb}\\
      &= \exists a,b,c,d \in \Sigma^* \mdot u \le cud \land cud \le acudb \implies u \le acudb & \pnote{def. of a subword; v=cud}\\
      &= u \le cud \land cud \le acudb \implies u \le acudb  & \pnote{$\exists-introduction$; a,b,c,d = a,b,c,d}\\
      &= true \land true \implies true & \pnote{def. of a subword}\\
      &= true\\
    \end{flalign*}

    So transitivity of $(\Sigma^*,\le)$ holds.\\
  

    Therefore, $(\Sigma^*,\le)$ is a weak partial order.\\

    \item Prove that $(\Sigma^*,\le)$ is not a weak total order.

    \emph{Proof.} To prove that $(\Sigma^*,\le)$ is not a weak total order, we must prove it is not total.\\
    \\
    \emph{Total:} $\ForallApp u \in \Sigma^* \mdot \exists v \in \Sigma^* \mdot u \le v \lor v \le u$

    \begin{flalign*}
      &\mname{For all } u \in \Sigma^*:\\
      &\phantom{{}=} \exists v \in \Sigma^* \mdot u \le v \lor v \le u\\
      &= u \le abc \lor abc \le u & \pnote{$\exists-introduction$; v=abc}\\
      &= false \lor false  & \pnote{def. of a subword}\\
      &= false\\
    \end{flalign*}

    So totality of $(\Sigma^*,\le)$ does not hold by proof by contradiction. Therefore, $(\Sigma^*,\le)$ is not a weak total order.\\


    \item Does $(\Sigma^*,\le)$ have a minimum element?  Justify your
      answer.

    A minimum element means that element is a subset of every other element in the set. $\epsilon$ is the minimum element, assuming that since $\epsilon$ denotes an empty word, $ab$, $ab\epsilon$, $\epsilon ab$, and $a\epsilon b$ are all equal words. Therefore, every word in $\Sigma^*$ contains an empty would, or $\epsilon$, which is why it is the minimum element.\\

    \item Does $(\Sigma^*,\le)$ have a maximum element?  Justify your
      answer.

    No, it does not have a maximum element. This is because the set of words $\Sigma^*$ is infinite. Since there is no limit on the length of the words, there are infinite combinations of words, meaning an infinite number of words in $\Sigma^*$. A maximum element means that all the other elements in the set are a subset of that element. Since the set is infinite, there cannot be a maximum element that contains all other elements. There cannot be an infinite element that contains infinite elements.\\

  \ee

  \item \textbf{[8 points]} Let $\Sigma = \set{a,b,c}$ be a finite
    alphabet.  Construct formulas that specify the following languages
    over $\Sigma$.

  \textbf{Aamina Hussain, hussaa54, February 14, 2021}\\
  \\
  Additional predicates:
  \begin{itemize}
    \item $\mname{between}(x,y) = z$ where $x, y, z \in \Sigma^*$ states that there is a word $z$ between the words $x$ and $y$. The word $z$ begins immediately after the end of the word $x$, and ends right before the beginning of the word $y$. For example, $\mname{between}(ab,cd) = xyz$ would be describing $abxyzcd$.
    \item $\mname{length}(x) = n$ where $x \in \Sigma^*$ and $n \in \mathbb{N}$ states the length of a word. For example, $\mname{length}(abc)=3$ would be true.
    \item The following is assuming the first position of the word is 1.
  \end{itemize}

  \be

    \item $\set{awa \mid w \in \Sigma^*}$.\\
    $\ForallApp x \mdot a(1) \land \mname{last}(x) \rightarrow a(x)$

    \item $\set{dwd \mid d \in \Sigma \text{ and } w \in \Sigma^*}$.\\
    $\ForallApp x \mdot (a(1) \land \mname{last}(x) \rightarrow a(x)) \lor (b(1) \land \mname{last}(x) \rightarrow b(x)) \lor (c(1) \land \mname{last}(x) \rightarrow c(x))$

    \item $\set{uaav \mid u, v \in \Sigma^*}$.\\
    $\ForallApp u, v \in \Sigma^* \mdot \mname{between}(u, v) = aa$
    
    \item $\set{uavbw \mid u,v,w \in \Sigma^*}$.\\
    $\ForallApp u, v, w \in \Sigma^* \mdot \mname{between}(u, v) = a \land \mname{between}(v, w) = b$

    \item $\Sigma^*$.\\
    $\ForallApp w \in \Sigma^* \mdot w$

    \item $\Sigma^* \setminus \set{\epsilon}$.\\
    $\ForallApp w \in \Sigma^* \mdot \lnot (\mname{length}(w)=1 \land \epsilon(1))$

    \item $\set{\epsilon}$.\\
    $\ForallApp w \in \Sigma^* \mdot \mname{length}(w)=1 \land \epsilon(1)$

    \item $\emptyset$.\\
    $\ForallApp w \in \Sigma^* \mdot \lnot w$

  \ee

\ee

\end{document}


