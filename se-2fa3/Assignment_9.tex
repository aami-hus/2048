\documentclass[11pt,fleqn]{article}

\setlength {\topmargin} {-.15in}
\setlength {\textheight} {8.6in}

\usepackage{amsmath}
\usepackage{amssymb}
\usepackage{amsthm}
\usepackage{url}
\usepackage{listings}
\usepackage{color}
\usepackage{tikz}
\usetikzlibrary{automata,positioning,arrows}
\usepackage{diagbox}
\usepackage{stackrel}

\setlength {\topmargin} {-.15in}
\setlength {\textheight} {8.6in}

\renewcommand{\labelenumi}{\theenumi.}
\renewcommand{\labelenumii}{\theenumii.}
\renewcommand{\labelenumiii}{\theenumiii.}
\newcommand{\be}{\begin{enumerate}}
	\newcommand{\ee}{\end{enumerate}}
\newcommand{\bi}{\begin{itemize}}
	\newcommand{\ei}{\end{itemize}}
\newcommand{\bc}{\begin{center}}
	\newcommand{\ec}{\end{center}}
\newcommand{\bsp}{\begin{sloppypar}}
	\newcommand{\esp}{\end{sloppypar}}
\newcommand{\mname}[1]{\mbox{\sf #1}}
\newcommand{\sB}{\mbox{$\cal B$}}
\newcommand{\sC}{\mbox{$\cal C$}}
\newcommand{\sF}{\mbox{$\cal F$}}
\newcommand{\sM}{\mbox{$\cal M$}}
\newcommand{\sP}{\mbox{$\cal P$}}
\newcommand{\sV}{\mbox{$\cal V$}}
\newcommand{\set}[1]{{\{ #1 \}}}
\newcommand{\Neg}{\neg}
\ifdefined \And
\renewcommand{\And}{\wedge}
\else
\newcommand{\And}{\wedge}
\fi
\newcommand{\Or}{\vee}
\newcommand{\Implies}{\Rightarrow}
\newcommand{\Iff}{\LeftRightarrow}
\newcommand{\Forall}{\forall}
\newcommand{\ForallApp}{\forall\,}
\newcommand{\Forsome}{\exists}
\newcommand{\ForsomeApp}{\exists\,}
\newcommand{\mdot}{\mathrel.}
\newcommand{\eps}{\epsilon}
\newcommand{\pnote}[1]{\langle \mbox{#1} \rangle}


\begin{document}

\begin{center}

{\large \textbf{COMPSCI/SFWRENG 2FA3}}\\[2mm]
{\large \textbf{Discrete Mathematics with Applications II}}\\[2mm]
{\large \textbf{Winter 2021}}\\[8mm]
{\huge \textbf{Assignment 9}}\\[6mm]
{\large \textbf{Dr.~William M. Farmer and Dr.~Mehrnoosh Askarpour}}\\[2mm]
{\large \textbf{McMaster University}}\\[6mm]
{\large Revised: March 28, 2021}

\end{center}

\medskip

Assignment 9 consists of two problems.  You must write your solutions
to the problems using LaTeX.

Please submit Assignment~9 as two files,
\texttt{Assignment\_9\_\emph{YourMacID}.tex} and
\texttt{Assignment\_9\_\emph{YourMacID}.pdf}, to the Assignment~9
folder on Avenue under Assessments/Assignments.
\texttt{\emph{YourMacID}} must be your personal MacID (written without
capitalization).  The \texttt{Assignment\_9\_\emph{YourMacID}.tex}
file is a copy of the LaTeX source file for this assignment
(\texttt{Assignment\_9.tex} found on Avenue under
Contents/Assignments) with your solution entered after each problem.
The \texttt{Assignment\_9\_\emph{YourMacID}.pdf} is the PDF output
produced by executing

\begin{itemize}

  \item[] \texttt{pdflatex Assignment\_9\_\emph{YourMacID}}

\end{itemize}

This assignment is due \textbf{Sunday, April 4, 2021 before
  midnight.}  You are allow to submit the assignment multiple times,
but only the last submission will be marked.  \textbf{Late submissions
  and files that are not named exactly as specified above will not be
  accepted!}  It is suggested that you submit your preliminary
\texttt{Assignment\_9\_\emph{YourMacID}.tex} and
\texttt{Assignment\_9\_\emph{YourMacID}.pdf} files well before the
deadline so that your mark is not zero if, e.g., your computer fails
at 11:50 PM on April 4.

\textbf{Although you are allowed to receive help from the
  instructional staff and other students, your submission must be your
  own work.  Copying will be treated as academic dishonesty! If any of
  the ideas used in your submission were obtained from other students
  or sources outside of the lectures and tutorials, you must
  acknowledge where or from whom these ideas were obtained.}

\newpage

\subsection*{Problems}

\be

  \item \textbf{[10 points]} Let $\Sigma = \set{a,b}$ and \[L = \set{x
    \in \Sigma^* \mid \#a(x) \text{ and } \#b(x) \text{ are both
      even}}.\] Construct a total Turing machine that accepts $L$.
    Present the TM using a transition table or diagram formally, and
    describe how it works informally.

  \bigskip

  \textbf{Aamina Hussain, hussaa54, April 4, 2021}

Let $M = (Q, \Sigma, \Gamma, \vdash, \textvisiblespace, \delta, s, t, r)$ be the TM where:

$Q = \set{s, q_1, q_2, q_3, t, r}$\\
$\Sigma = \set{a, b}$\\
$\Gamma = \Sigma \; \cup$ \{$\vdash,$ \textvisiblespace\}\\
$\delta$ is defined by the following table:

\bc
\begin{table}[hbt!]
\centering
\begin{tabular}{l|cccc}
      & $\vdash$         & $a$             & $b$             & \textvisiblespace            \\
\hline
$s$   & $(s, \vdash, R)$ & $(q_3, a, R)$   & $(q_1, b, R)$   & $(t, \--, \--)$  \\
$q_1$ & $(r, \vdash, R)$ & $(q_2, a, R)$   & $(s, b, R)$     & $(r, \--, \--)$  \\
$q_2$ & $(r, \vdash, R)$ & $(q_1, a, R)$   & $(q_3, b, R)$   & $(r, \--, \--)$  \\
$q_3$ & $(r, \vdash, R)$ & $(s, a, R)$     & $(q_2, b, R)$   & $(r, \--, \--)$  \\
$t$   & $(t, \vdash, R)$ & $(t, \--, \--)$ & $(t, \--, \--)$ & $(t, \--, \--)$  \\
$r$   & $(r, \vdash, R)$ & $(r, \--, \--)$ & $(r, \--, \--)$ & $(r, \--, \--)$ 
\end{tabular}
\end{table}
\ec

This TM only moves right, starting from the left endmarker, and every time it reads from the input tape, it goes to the corresponding state. State $s$ is when there are an even number of a's and b's. State $q_1$ is when there are an odd number of b's but an even number of a's. State $q_2$ is when there are an odd number of a's and an odd number of b's. State $q_3$ is when there are an odd number of a's and an even number of b's. You can only transition to the acceptance state $t$ if you are in state $s$ when you read the first blank symbol \textvisiblespace.

\clearpage
  \item \textbf{[10 points]} Let $\Sigma = \set{a,b}$ and \[L = \set{x
    \in \Sigma^* \mid \#a(x) = \#b(x)}.\] Construct a total Turing
    machine that accepts $L$.  Present the TM using a transition table
    or diagram formally, and describe how it works informally.

  \bigskip

  \textbf{Aamina Hussain, hussaa54, April 4, 2021}

Let $M = (Q, \Sigma, \Gamma, \vdash, \textvisiblespace, \delta, s, t, r)$ be the TM where:

$Q = \set{s, q_1, q_2, q_3, t, r}$\\
$\Sigma = \set{a, b}$\\
$\Gamma = \Sigma \; \cup$ \{$\vdash,$ \textvisiblespace, $x$\}\\
$\delta$ is defined by the following table:

  \bc
  \begin{table}[h]
  \centering
  \begin{tabular}{l|ccccc}
        & $\vdash$         & $a$             & $b$             & $x$             & \textvisiblespace\\
  \hline
  $s$   & $(s, \vdash, R)$ & $(q_1, x, R)$   & $(q_2, x, R)$   & $(s, x, R)$     & $(t, \--, \--)$  \\
  $q_1$ & $(r, \vdash, R)$ & $(q_1, a, R)$   & $(q_3, x, L)$   & $(q_1, x, R)$   & $(r, \--, \--)$  \\
  $q_2$ & $(r, \vdash, R)$ & $(q_3, x, L)$   & $(q_2, b, R)$   & $(q_2, x, R)$   & $(r, \--, \--)$  \\
  $q_3$ & $(s, \vdash, R)$ & $(q_3, a, L)$   & $(q_3, b, L)$   & $(q_3, x, L)$   & $(r, \--, \--)$  \\
  $t$   & $(t, \vdash, R)$ & $(t, \--, \--)$ & $(t, \--, \--)$ & $(t, \--, \--)$ & $(t, \--, \--)$  \\
  $r$   & $(r, \vdash, R)$ & $(r, \--, \--)$ & $(r, \--, \--)$ & $(r, \--, \--)$ & $(r, \--, \--)$ 
  \end{tabular}
  \end{table}
  \ec

\begin{enumerate}
\item Begin in the start state $s$.
\item If there is an $a$ available before a $b$ on the input tape: Move right until you see the first a, replace it with an x, and transition into state $q_1$. Then continue moving right until you see the first b, replace it with an x, and transition into state $q_3$. Then move left until you reach the left endmarker and transition into state $s$.
\item If there is a $b$ available before an $a$ on the input tape: Move right until you see the first b, replace it with an x, and transition into state $q_2$. Then continue moving right until you see the first a, replace it with an x, and transition into state $q_3$. Then move left until you reach the left endmarker and transition into state $s$.
\item Repeat step b)/step c) until all the a's and b's are replaced with x's.
\item You can only transition to the acceptance state $t$ if you are in state $s$ when you read the first blank symbol \textvisiblespace.
\end{enumerate}


\ee

\end{document}


