\documentclass[11pt,fleqn]{article}

\setlength {\topmargin} {-.15in}
\setlength {\textheight} {8.6in}

\usepackage{amsmath}
\usepackage{amssymb}
\usepackage{amsthm}
\usepackage{color}

\renewcommand{\labelenumii}{\theenumii.}

\newcommand{\mname}[1]{\mbox{\sf #1}}
\newcommand{\pnote}[1]{{\langle \text{#1} \rangle}}

\begin{document}

\begin{center}

  {\large \textbf{COMPSCI/SFWRENG 2FA3}}\\[2mm]
  {\large \textbf{Discrete Mathematics with Applications II}}\\[2mm]
  {\large \textbf{Winter 2021}}\\[8mm]
  {\huge \textbf{Assignment 1}}\\[6mm]
  {\large \textbf{Dr.~William M. Farmer and Dr.~Mehrnoosh Askarpour}}\\[2mm]
  {\large \textbf{McMaster University}}\\[6mm]
  {\large Revised: January 21, 2021}

\end{center}

\medskip

Assignment 1 consists of some background definitions, two sample
problems, and two required problems.  You must write your solutions to
the required problems using LaTeX.  Use the solutions of the sample
problems as a guide.

Please submit Assignment~1 as two files,
\texttt{Assignment\_1\_\emph{YourMacID}.tex} and
\texttt{Assignment\_1\_\emph{YourMacID}.pdf}, to the Assignment~1
folder on Avenue under Assessments/Assignments.
\texttt{\emph{YourMacID}} must be your personal MacID (written without
capitalization).  The \texttt{Assignment\_1\_\emph{YourMacID}.tex}
file is a copy of the LaTeX source file for this assignment
(\texttt{Assignment\_1.tex} found on Avenue under
Contents/Assignments) with your solution entered after each required
problem.  The \texttt{Assignment\_1\_\emph{YourMacID}.pdf} is the PDF
output produced by executing

\begin{itemize}

  \item[] \texttt{pdflatex Assignment\_1\_\emph{YourMacID}}

\end{itemize}

This assignment is due \textbf{Sunday, January 31, 2021 before
  midnight.}  You are allow to submit the assignment multiple times,
but only the last submission will be marked.  \textbf{Late submissions
  and files that are not named exactly as specified above will not be
  accepted!}  It is suggested that you submit your preliminary
\texttt{Assignment\_1\_\emph{YourMacID}.tex} and
\texttt{Assignment\_1\_\emph{YourMacID}.pdf} files well before the
deadline so that your mark is not zero if, e.g., your computer fails
at 11:50 PM on January 31.

\textbf{Although you are allowed to receive help from the
  instructional staff and other students, your submission must be your
  own work.  Copying will be treated as academic dishonesty! If any of
  the ideas used in your submission were obtained from other students
  or sources outside of the lectures and tutorials, you must
  acknowledge where or from whom these ideas were obtained.}

\subsection*{Background}

\begin{enumerate}

  \item The notation $\sum^{n}_{i=m}f(i)$ is defined by: 
    \[\sum^{n}_{i=m}f(i) =
      \left\{\begin{array}{ll}
               0                            & \textrm{if } m > n\\
               \big(\sum^{n-1}_{i=m}f(i)\big) + f(n) & \textrm{if } m \le n
             \end{array}
      \right.\]

  \item The notation $\prod^{n}_{i=m}f(i)$ is defined by: 
    \[\prod^{n}_{i=m}f(i) =
      \left\{\begin{array}{ll}
               1                             & \textrm{if } m > n\\
               \big(\prod^{n-1}_{i=m}f(i)\big) * f(n) & \textrm{if } m \le n
             \end{array}
      \right.\] 

  \item The factorial function $\mname{fact} : \mathbb{N} \rightarrow
    \mathbb{N}$ is defined by:
    \[\mname{fact}(n) = 
      \left\{\begin{array}{ll}
               1 & \textrm{if } n = 0 \\
               \mname{fact}(n-1) * n & \textrm{if } n > 0 
             \end{array}
      \right.\]

  \item The Fibonacci sequence $\mname{fib} : \mathbb{N} \rightarrow
    \mathbb{N}$ is defined by:
    \[\mname{fib}(n) = 
      \left\{\begin{array}{ll}
               0 & \textrm{if } n = 0 \\
               1 & \textrm{if } n = 1 \\
               \mname{fib}(n-1) + \mname{fib}(n-2) & \textrm{if } n \ge 2
             \end{array}
      \right.\]

\end{enumerate}

\subsection*{Sample Problems}

\begin{enumerate}

  \item Prove $\sum^{n-1}_{i=0}2^i = 2^n - 1$ for all $n \in
    \mathbb{N}$.

\begin{proof}
Let $P(n) \equiv \sum^{n-1}_{i=0}2^i = 2^n - 1$.  We will prove $P(n)$
for all $n \in \mathbb{N}$ by weak induction.

\medskip

\emph{Base case}: $n = 0$.  We must show $P(0)$.
\begin{align*}
  &\phantom{{}=} \sum_{i=0}^{0-1} 2^{i} & \pnote{LHS of $P(0)$}\\
  &= \sum^{-1}_{i = 0} 2^i  & \pnote{arithmetic}\\
  &= 0          & \pnote{definition of $\textstyle{\sum}_{i=m}^{n} f(i)$ when $m > n$}\\
  &= 1 - 1      & \pnote{arithmetic}\\
  &= 2^0 - 1    & \pnote{arithmetic; RHS of $P(0)$}
\end{align*}
So $P(0)$ holds.

\medskip

\emph{Induction step}: $n \ge 0$. Assume $P(n)$. We must show $P(n + 1)$.
\begin{align*}
  &\phantom{{}=} \sum_{i=0}^{(n + 1) - 1} 2^i   & \pnote{LHS of $P(n + 1)$}\\
  &= \sum_{i=0}^{n} 2^{i}       & \pnote{arithmetic}\\
  &= \big(\textstyle{\sum}_{i=0}^{n-1} 2^i\big) + 2^n & \pnote{definition of $\textstyle{\sum}_{i=m}^{n} f(i)$}\\
  &= (2^n - 1) + 2^n           & \pnote{induction hypothesis: $P(n)$}\\
  &= 2*2^n - 1                 & \pnote{arithmetic}\\
  &= 2^{n+1} - 1               & \pnote{arithmetic; RHS of $P(n + 1)$}
\end{align*}
So $P(n + 1)$ holds.

\medskip

Therefore, $P(n)$ holds for all $n \in \mathbb{N}$ by weak induction.
\end{proof}

  \item Prove that, if $n \in \mathbb{N}$ with $n \ge 2$, then $n$ is
    a prime number or a product of prime numbers.

\begin{proof}
Let $P(n)$ hold iff $n$ is a product of prime numbers.  We will prove
$P(n)$ for all $n \in \mathbb{N}$ with $n \ge 2$ by strong induction.

\medskip

\emph{Base case}: $n = 2$.  We must show $P(2)$.  Since 2 is a prime
number, $P(2)$ obviously holds.

\medskip

\emph{Induction step}: $n > 2$.  Assume $P(2), P(3), \ldots, P(n-1)$
hold.  We must show $P(n)$.

\emph{Case 1}: $n$ is a prime number.  Then $P(n)$ obviously holds.

\emph{Case 2}: $n$ is not a prime number.  Then $n = x * y$ where $x,y
\in \mathbb{N}$ with $2 \le x,y \le n - 1$.  Thus, by the induction
hypothesis ($P(x)$ and $P(y)$), \[x = p_0 * \cdots *p_i\] and \[y =
q_0 * \cdots *q_j\] where $p_0,\ldots,p_i,q_0,\ldots,q_j$ are prime
numbers.  Then \[n = x * y = p_0 * \cdots * p_i * q_0 * \cdots * q_j\]
and so $P(n)$ holds since $n$ is a product of prime numbers.

\medskip

Therefore, $P(n)$ holds for all $n \in \mathbb{N}$ with $n \ge 2$ by
strong induction.
\end{proof}

\end{enumerate}

\subsection*{Required Problems}

\begin{enumerate}

  \item \textbf{[10 points]} Prove \[\sum^{n}_{i = 0} i *
    \mname{fact}(i) = \mname{fact}(n+1) - 1\] for all $n \in
    \mathbb{N}$.

  \bigskip

  \textcolor{blue}{\textbf{Put your name, MacID, and date here.}}

  \textcolor{blue}{\textbf{Put your solution here.}}

  \bigskip

  \item \textbf{[10 points]} Prove that, for all $n \in \mathbb{N}$,
    $\mname{fib}(n)$ is even if $n = 3k$ for some $k \in \mathbb{N}$,
    is odd if $n = 3k + 1$ for some $k \in \mathbb{N}$, and is odd if
    $n = 3k + 2$ for some $k \in \mathbb{N}$.

  \bigskip

  \textcolor{blue}{\textbf{Put your name, MacID, and date here.}}

  \textcolor{blue}{\textbf{Put your solution here.}}

\end{enumerate}

\end{document}


