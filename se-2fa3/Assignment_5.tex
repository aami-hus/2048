\documentclass[11pt,fleqn]{article}

\setlength {\topmargin} {-.15in}
\setlength {\textheight} {8.6in}

\usepackage{amsmath}
\usepackage{amssymb}
\usepackage{amsthm}
\usepackage{url}
\usepackage{color}
\usepackage{tikz}
\usepackage{diagbox}
\usetikzlibrary{automata,positioning,arrows}

\setlength {\topmargin} {-.15in}
\setlength {\textheight} {8.6in}

\renewcommand{\labelenumi}{\theenumi.}
\renewcommand{\labelenumii}{\theenumii.}
\renewcommand{\labelenumiii}{\theenumiii.}
\newcommand{\be}{\begin{enumerate}}
	\newcommand{\ee}{\end{enumerate}}
\newcommand{\bi}{\begin{itemize}}
	\newcommand{\ei}{\end{itemize}}
\newcommand{\bc}{\begin{center}}
	\newcommand{\ec}{\end{center}}
\newcommand{\bsp}{\begin{sloppypar}}
	\newcommand{\esp}{\end{sloppypar}}
\newcommand{\mname}[1]{\mbox{\sf #1}}
\newcommand{\sB}{\mbox{$\cal B$}}
\newcommand{\sC}{\mbox{$\cal C$}}
\newcommand{\sF}{\mbox{$\cal F$}}
\newcommand{\sM}{\mbox{$\cal M$}}
\newcommand{\sP}{\mbox{$\cal P$}}
\newcommand{\sV}{\mbox{$\cal V$}}
\newcommand{\set}[1]{{\{ #1 \}}}
\newcommand{\Neg}{\neg}
\ifdefined \And
\renewcommand{\And}{\wedge}
\else
\newcommand{\And}{\wedge}
\fi
\newcommand{\Or}{\vee}
\newcommand{\Implies}{\Rightarrow}
\newcommand{\Iff}{\LeftRightarrow}
\newcommand{\Forall}{\forall}
\newcommand{\ForallApp}{\forall\,}
\newcommand{\Forsome}{\exists}
\newcommand{\ForsomeApp}{\exists\,}
\newcommand{\mdot}{\mathrel.}
\newcommand{\eps}{\epsilon}
\newcommand{\pnote}[1]{\langle \mbox{#1} \rangle}


\begin{document}
	
	\begin{center}
		
		{\large \textbf{COMPSCI/SFWRENG 2FA3}}\\[2mm]
		{\large \textbf{Discrete Mathematics with Applications II}}\\[2mm]
		{\large \textbf{Winter 2021}}\\[8mm]
		{\huge \textbf{Assignment 5}}\\[6mm]
		{\large \textbf{Dr.~William M. Farmer and Dr.~Mehrnoosh Askarpour}}\\[2mm]
		{\large \textbf{McMaster University}}\\[6mm]
		{\large Revised: March 2, 2021}
		
	\end{center}
	
	\medskip
	
	Assignment 5 consists of a problem with three subproblems.  You must write your solutions
	to the problems using LaTeX.
	
	Please submit Assignment~5 as two files,
	\texttt{Assignment\_5\_\emph{YourMacID}.tex} and
	\texttt{Assignment\_5\_\emph{YourMacID}.pdf}, to the Assignment~5
	folder on Avenue under Assessments/Assignments.
	\texttt{\emph{YourMacID}} must be your personal MacID (written without
	capitalization).  The \texttt{Assignment\_5\_\emph{YourMacID}.tex}
	file is a copy of the LaTeX source file for this assignment
	(\texttt{Assignment\_5.tex} found on Avenue under
	Contents/Assignments) with your solution entered after each problem.
	The \texttt{Assignment\_5\_\emph{YourMacID}.pdf} is the PDF output
	produced by executing
	
	\begin{itemize}
		
		\item[] \texttt{pdflatex Assignment\_5\_\emph{YourMacID}}
		
	\end{itemize}
	
	This assignment is due \textbf{Sunday, March 7, 2021 before
		midnight.}  You are allow to submit the assignment multiple times,
	but only the last submission will be marked.  \textbf{Late submissions
		and files that are not named exactly as specified above will not be
		accepted!}  It is suggested that you submit your preliminary
	\texttt{Assignment\_5\_\emph{YourMacID}.tex} and
	\texttt{Assignment\_5\_\emph{YourMacID}.pdf} files well before the
	deadline so that your mark is not zero if, e.g., your computer fails
	at 11:50 PM on March 7.
	
	\textbf{Although you are allowed to receive help from the
		instructional staff and other students, your submission must be your
		own work.  Copying will be treated as academic dishonesty! If any of
		the ideas used in your submission were obtained from other students
		or sources outside of the lectures and tutorials, you must
		acknowledge where or from whom these ideas were obtained.}
	
	\newpage
	
	\subsection*{Presenting DFAs and NFAs Transition Diagrams}
	
	In this assignment you are asked to present DFAs as transition
	diagrams.  You are can do this in one of two ways.
	
	The first way is to present the diagram using the LaTeX graphics
	package TikZ.  The TikZ code can either be written by hand or
	automatically generated using the finsm system available at
	\texttt{http:finsm.io}.
	
	Here are some examples of how it can be used to create
	DFA and NFA transition diagrams that appear in the lectures slides:
	
	\begin{center}
		\begin{tikzpicture}[shorten >=1pt,node distance=2.5cm,on grid,auto] 
			\node[state, initial, thick] (q_0)   {$q_0$}; 
			\node[state, thick] (q_1) [right=of q_0] {$q_1$}; 
			\node[state, thick] (q_2) [right=of q_1] {$q_2$}; 
			\node[state, accepting, thick] (q_3) [right=of q_2] {$q_3$};
			\path[->, thick, >=stealth] 
			(q_0) edge [bend left, above] node {$a$} (q_1)
			edge [loop, above] node {$b$} (q_2)
			(q_1) edge [bend left, above] node {$a$} (q_2)
			edge [bend left, above] node {$b$} (q_0)
			(q_2) edge [bend left, above] node {$a$} (q_3) 
			edge [bend left, below]  node {$b$} (q_0)
			(q_3) edge [loop, above] node {$a,b$} (q_2); 
		\end{tikzpicture}
	\end{center}
	
	\begin{center}
		\begin{tikzpicture}[shorten >=1pt,node distance=4cm,on grid,auto] 
			\node[state, initial, accepting, thick] (q_0)   {$q_0$}; 
			\node[state, thick] (q_1) [right=of q_0] {$q_1$}; 
			\node[state, thick] (q_2) [below=of q_0] {$q_2$}; 
			\node[state, thick] (q_3) [right=of q_2] {$q_3$};
			\path[->, thick, >=stealth] 
			(q_0) edge [bend left, above] node {1} (q_1)
			edge [bend left, right] node {0} (q_2)
			(q_1) edge [bend left, below] node {1} (q_0)
			edge [bend left, right] node {0} (q_3)
			(q_2) edge [bend left, above] node {1} (q_3) 
			edge [bend left, left]  node {0} (q_0)
			(q_3) edge [bend left, below] node {1} (q_2) 
			edge [bend left, left]  node {0} (q_1);
		\end{tikzpicture}
	\end{center}
	
	\begin{center}
		\begin{tikzpicture}[shorten >=1pt,node distance=1.7cm,on grid,auto] 
			\node[state, initial, thick] (q_0)   {$q_0$}; 
			\node[state, thick] (q_1) [right=of q_0] {$q_1$}; 
			\node[state, thick] (q_2) [right=of q_1] {$q_2$}; 
			\node[state, thick] (q_3) [right=of q_2] {$q_3$};
			\node[state, thick] (q_4) [right=of q_3] {$q_4$};
			\node[state, thick, accepting] (q_5) [right=of q_4] {$q_5$};
			\path[->, thick, >=stealth] 
			(q_0) edge [loop, above] node {0,1} (q_1)
			edge [right, above] node {1} (q_1)
			(q_1) edge [right, above] node {0,1} (q_2)
			(q_2) edge [right, above] node {0,1} (q_3)
			(q_3) edge [right, above] node {0,1} (q_4)
			(q_4) edge [right, above] node {0,1} (q_5);
		\end{tikzpicture}
	\end{center}
	
	The second way is to take a picture of a hand-written transition
	diagram and then embed it into your assignment using the following
	LaTeX code:
	\begin{verbatim}
		\begin{center}
			\includegraphics[scale = 0.5]{diagram.jpg}
		\end{center}
	\end{verbatim}
	Please make sure your diagram is legible.
	
	\subsection*{Problems}
	Let $\Sigma = \{a, b, c\}$ and $L =\{x \in
        \Sigma^* \mid x \text{ contains ``cc"}\}$: \be
	
	\item \textbf{[7 points]} Construct an NFA that accepts L.
	\item \textbf{[8 points]} Construct the equivalent DFA using the subset construction.
	\item \textbf{[5 points]} Is the constructed DFA minimal? If not, construct the equivalent minimal DFA.

        \ee

        \noindent
        Present your FAs using transition diagrams.
        
        \bigskip

        \noindent
        \textbf{Aamina Hussain, hussaa54, March 7, 2021}

        \medskip

        \be

        \item The NFA transition diagram is below.

        \begin{center}
		\begin{tikzpicture}[]
		    \node[initial,thick,state] at (-5.6875,0.9625) (06a97fcb) {$q_{1}$};
		    \node[thick,state] at (-1.1375,0.9875) (5490009c) {$q_{2}$};
		    \node[thick,accepting,state] at (3.1875,0.9625) (1186a718) {$q_{3}$};
		    \path[->, thick, >=stealth]
		    (06a97fcb) edge [below,in = -180, out = 0] node {$c$} (5490009c)
		    (06a97fcb) edge [loop,min distance = 1.25cm,above,in = 121, out = 59] node {$a,b,c$} (06a97fcb)
		    (5490009c) edge [below,in = 180, out = 0] node {$c$} (1186a718)
		    (1186a718) edge [loop,min distance = 1.25cm,above,in = 121, out = 59] node {$a,b,c$} (1186a718)
		    ;
		\end{tikzpicture}
		\end{center}

        \item The equivalent DFA of the above NFA is below.

        \begin{center}
		\begin{tikzpicture}[]
		    \node[initial,thick,state] at (-3.3875,1.8625) (12778f70) {$\{q_{1}\}$};
		    \node[thick,state] at (1.8125,1.8625) (eeba0637) {$\{q_{1}, q_{2}\}$};
		    \node[thick,accepting,state] at (-3.4625,-1.6875) (3c7dea05) {$\{q_{1}, q_{3}\}$};
		    \node[thick,accepting,state] at (1.8125,-1.6875) (4420802f) {$\{q_{1}, q_{2}, q_{3}\}$};
		    \path[->, thick, >=stealth]
		    (12778f70) edge [loop,min distance = 1.25cm,above,in = 121, out = 59] node {$a,b$} (12778f70)
		    (12778f70) edge [above,in = 179, out = 1] node {$c$} (eeba0637)
		    (eeba0637) edge [below,in = -18, out = -162] node {$a,b$} (12778f70)
		    (eeba0637) edge [left,in = 90, out = -90] node {$c$} (4420802f)
		    (3c7dea05) edge [below,in = 180, out = 0] node {$c$} (4420802f)
		    (3c7dea05) edge [loop,min distance = 1.25cm,above,in = 121, out = 59] node {$a,b$} (3c7dea05)
		    (4420802f) edge [loop,min distance = 1.33cm,below,in = -56, out = -120] node {$c$} (4420802f)
		    (4420802f) edge [above,in = 15, out = 164] node {$a,b$} (3c7dea05)
		    ;
		\end{tikzpicture}
		\end{center}
		
		Transition Table:

		\begin{table}
		\centering
		\begin{tabular}{cc|c|c|c}
		                         & \diagbox{{}$P(Q)$}{{}$\Sigma$} & a            & b            & c                  \\ 
		\cline{2-5}
		                         & $\{\emptyset\}$                      & $\{\emptyset\}$  & $\{\emptyset\}$  & $\{\emptyset\}$        \\
		start $\rightarrow$      & $\{q_1\}$                            & $\{q_1\}$      & $\{q_1\}$      & $\{q_1, q_2\}$       \\
		                         & $\{q_2\}$                            & $\{\emptyset\}$  & $\{\emptyset\}$  & $\{q_3\}$            \\
		final $\rightarrow$ & $\{q_3\}$                            & $\{q_3\}$      & $\{q_3\}$      & $\{q_3\}$            \\
		                         & $\{q_1, q_2\}$                       & $\{q_1\}$      & $\{q_1\}$      & $\{q_1, q_2, q_3\}$  \\
		final $\rightarrow$ & $\{q_1, q_3\}$                       & $\{q_1, q_3\}$ & $\{q_1, q_3\}$ & $\{q_1, q_2, q_3\}$  \\
		final $\rightarrow$ & $\{q_2, q_3\}$                       & $\{q_3\}$      & $\{q_3\}$      & $\{q_3\}$            \\
		final $\rightarrow$ & $\{q_1, q_2, q_3\}$                  & $\{q_1, q_3\}$ & $\{q_1, q_3\}$ & $\{q_1, q_2, q_3\}$ 
		\end{tabular}
		\end{table}
		
		The unreachable states are $\{\emptyset\}$, $\{q_2\}$, $\{q_3\}$, and $\{q_2, q_3\}$.

        \item The DFA in question 2 is not minimal. This is because there are some equivalent states as you can see in the transition table above. States $\{q_1, q_3\}$ and $\{q_1, q_2, q_3\}$ are equivalent. The states they transition to in each case (for input a, b, and c) are the same. Removing the state $\{q_1, q_3\}$ would result in the same strings being accepted or rejected. Therefore, the new DFA is equivalent and minimal.

        The equivalent minimal DFA is below.

        \begin{center}
		\begin{tikzpicture}[]
		    \node[initial,thick,state] at (-3.3875,1.8625) (e1259b42) {$\{q_{1}\}$};
		    \node[thick,state] at (0.8375,1.8375) (6783c5c2) {$\{q_{1}, q_{2}\}$};
		    \node[thick,accepting,state] at (4.9625,1.8625) (546a00cc) {$\{q_{1}, q_{2}, q_{3}\}$};
		    \path[->, thick, >=stealth]
		    (e1259b42) edge [loop,min distance = 1.25cm,above,in = 121, out = 59] node {$a,b$} (e1259b42)
		    (e1259b42) edge [above,in = 179, out = 1] node {$c$} (6783c5c2)
		    (6783c5c2) edge [below,in = -22, out = -159] node {$a,b$} (e1259b42)
		    (6783c5c2) edge [below,in = -180, out = 0] node {$c$} (546a00cc)
		    (546a00cc) edge [loop,min distance = 1.28cm,above,in = 122, out = 60] node {$a,b,c$} (546a00cc)
		    ;
		\end{tikzpicture}
		\end{center}

        \ee
        

\end{document}


